\documentclass[12pt, a4paper]{article}
\usepackage[spanish]{babel}
\usepackage{amsmath}
\usepackage{booktabs}
\usepackage{multirow}
\usepackage{graphicx}
\usepackage[margin=2cm]{geometry}

\title{Fórmulas Clave en Arquitectura de Computadores}
\author{Resumen para Estudio}
\date{\today}

\begin{document}

\maketitle

\section{Fórmulas Esenciales}

\subsection{Ciclos de Reloj}
\[
\text{Ciclos totales} = \sum (\text{N° instrucciones tipo } X \times \text{CPI de } X)
\]
\textbf{Ejemplo:} 
\begin{itemize}
    \item 500 inst. aritméticas (CPI=1): \(500 \times 1 = 500\) ciclos.
    \item 50 inst. almacenamiento (CPI=5): \(50 \times 5 = 250\) ciclos.
\end{itemize}

\subsection{Tiempo de Ejecución}
\[
T = \frac{\text{Ciclos totales}}{\text{Frecuencia de reloj}} \quad \text{(Frecuencia en Hz)}
\]
\textbf{Unidades:}
\begin{itemize}
    \item 1 GHz = \(10^9\) Hz.
    \item 1 \(\mu\)s = \(10^{-6}\) s.
\end{itemize}

\subsection{CPI Global}
\[
\text{CPI} = \frac{\text{Ciclos totales}}{\text{N° total de instrucciones}} = \sum (\% \text{inst. } X \times \text{CPI } X)
\]
\textbf{Ejemplo:} 
\[
\text{CPI} = (0.1 \times 1) + (0.2 \times 2) + \dots
\]

\subsection{Aceleración (\textit{Speedup})}
\[
\text{Speedup} = \frac{T_{\text{original}}}{T_{\text{mejorado}}}
\]
\textbf{Relacionado:}
\[
T_{\text{mejorado}} = T_{\text{original}} \times (1 - \text{\% reducción})
\]

\subsection{Frecuencia Requerida}
\[
f_{\text{nuevo}} = f_{\text{original}} \times \frac{\text{CPI}_{\text{nuevo}}}{\text{CPI}_{\text{original}}}} \times \frac{T_{\text{original}}}{T_{\text{nuevo}}}
\]

\subsection{Instrucciones para un Tiempo Objetivo}
\[
I = \text{IPC} \times f \times T \quad \text{(IPC} = 1/\text{CPI)}
\]

\subsection{IPC (Instrucciones por Ciclo)}
\[
\text{IPC} = \frac{1}{\text{CPI}} = \frac{\text{N° instrucciones}}{\text{Ciclos totales}}
\]

\section{Estrategias Clave}

\begin{itemize}
    \item \textbf{Identificar el problema:} ¿Calcular tiempo, CPI, ciclos o optimización?
    \item \textbf{Unidades:} Siempre convertir GHz a Hz (\( \times 10^9\)) y verificar segundos vs. \(\mu\)s.
    \item \textbf{Optimización:} Reducir instrucciones con alto CPI (ej: carga/almacenamiento) mejora rendimiento.
    \item \textbf{Speedup:} Comparar \(T_{\text{original}}\) vs. \(T_{\text{mejorado}}\).
\end{itemize}

\section{Ejemplo Resuelto}

\subsection{Datos Iniciales}
\begin{tabular}{lcc}
    \toprule
    Tipo Instrucción & N° Instrucciones & CPI \\
    \midrule
    Aritméticas      & 500             & 1   \\
    Carga           & 100             & 5   \\
    Almacenamiento  & 50              & 5   \\
    Saltos          & 50              & 2   \\
    \bottomrule
\end{tabular}

\subsection{Pasos}
\begin{enumerate}
    \item \textbf{Ciclos totales:} 
    \[
    500 \times 1 + 100 \times 5 + 50 \times 5 + 50 \times 2 = 1350 \text{ ciclos}
    \]
    \item \textbf{CPI:} 
    \[
    \frac{1350}{700} \approx 1.93
    \]
    \item \textbf{Tiempo (2 GHz):} 
    \[
    \frac{1350}{2 \times 10^9} = 0.675 \mu\text{s}
    \]
\end{enumerate}

\section{Resumen en Tabla}
\begin{tabular}{ll}
    \toprule
    \textbf{Concepto} & \textbf{Fórmula} \\
    \midrule
    Ciclos totales & \(\sum (\text{Inst.} \times \text{CPI})\) \\
    Tiempo & \(\frac{\text{Ciclos}}{f}\) \\
    CPI & \(\frac{\text{Ciclos}}{\text{Instrucciones}}\) \\
    Speedup & \(\frac{T_{\text{original}}}{T_{\text{mejorado}}}\) \\
    \bottomrule
\end{tabular}

\end{document}